%%%%%%%%%%%%%%%%%%%%%%%%%%%%%%%%%%%%%%
%
% This is how you write code:
%
% \begin{minted}{matlab}
% foo = [2 1 0;1 4 3;2 4.5 6];
% \end{minted}
%

% This is how you import code:
% 
% \inputminted[linenos]{matlab}{foo_bar.m}
%
 
% Most figures are imported this way:
%
% \begin{figure}
% \includegraphics[width=\textwidth]{foo_figure}
% \caption{This is a caption}
% \end{figure}
%
%%%%%%%%%%%%%%%%%%%%%%%%%%%%%%%%%%%%%%


\documentclass[00-main.tex]{subfiles}
\begin{document}



\subsection*{b.}
Let $H$ be an $n \times n$ Hessenberg matrix. An upper Hessenberg matrix is a matrix whose elements below its first subdiagonal are zero, and an upper Hessenberg matrix is a matrix whose elements above its first superdiagonal is zero.The elements to be 'zeroed out' are in the subdiagonal of $H$ if $H$ is in upper triangular form, and in the superdiagonal of $H$ if $H$ is in lower triangular form. 
$H$ is thus at most $n-1$ Givens rotations away from becoming a triangular matrix, since the number of elements in the first sub- or superdiagonal is $n-1$. 

\subsubsection*{Procedure for computing the $QR$-decomposition of an upper Hessenberg matrix using Givens rotations}
\begin{algorithmic}
\Function{compute-QR}{x,y}
\State \# For all elements in $H$ that needs to be zeroed out 
\For{$k = 1,2, ..., n-1$}
  \State \# Calculate the Givens matrix for 
  \State \# zeroing out the element in $H(k+1,k)$ 
  \State $[c_k, s_k] \gets \textsc{calculate-givens}(H[k,k], H[k+1,k])$
  \State
  \State \# Apply the Givens rotation from the left
  \State $ H \gets 
  			 \left[ \begin{array}{c c}
      			c_k & s_k \\
     			-s_k & c_k
  			 \end{array} \right]
		   H
		$
\EndFor
\For{$k = 1,2, ..., n-1$}
  \State \# Apply the Givens rotation from the right
  \State $ H \gets H 
  			\left[ \begin{array}{c c}
      			c_k & s_k \\
     			-s_k & c_k
  			\end{array} \right]
		$
\EndFor
\EndFunction
\end{algorithmic}

\subsubsection*{Sub-procedure for calculating the givens elements $s$ and $c$}
\begin{algorithmic}
\Function{calculate-givens}{x,y}
\State $r \gets \sqrt[]{x^2+y^2 }$ 
\State $c \gets \frac{x}{r} $
\State $s \gets \frac{y}{r} $
\State \Return $[c, s]$
\EndFunction
\end{algorithmic}


%\bibliosub
\end{document}