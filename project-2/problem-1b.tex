%%%%%%%%%%%%%%%%%%%%%%%%%%%%%%%%%%%%%%
%
% This is how you write code:
%
% \begin{minted}{matlab}
% foo = [2 1 0;1 4 3;2 4.5 6];
% \end{minted}
%

% This is how you import code:
% 
% \inputminted[linenos]{matlab}{foo_bar.m}
%
 
% Most figures are imported this way:
%
% \begin{figure}
% \includegraphics[width=\textwidth]{foo_figure}
% \caption{This is a caption}
% \end{figure}
%
%%%%%%%%%%%%%%%%%%%%%%%%%%%%%%%%%%%%%%


\documentclass[00-main.tex]{subfiles}
\begin{document}



\subsection*{b.}
Let $H$ be an $n \times n$ Hessenberg matrix. An upper Hessenberg matrix is a matrix whose elements below its first subdiagonal are zero, and an upper Hessenberg matrix is a matrix whose elements above its first superdiagonal is zero. The elements to be 'zeroed out' are in the subdiagonal of $H$ if $H$ is in upper triangular form, and in the superdiagonal of $H$ if $H$ is in lower triangular form. 
$H$ is thus at most $n-1$ Givens rotations away from becoming a triangular matrix, since the number of elements in the first sub- or superdiagonal is $n-1$. 
In the following it is assumed that $H$ is in upper form.

\subsubsection*{Procedure for computing the $QR$-decomposition of an upper Hessenberg matrix using Givens rotations}
\begin{algorithmic}
\Function{compute-QR}{x,y}
\State // For all elements in $H$ that needs to be zeroed out 
\For{$k = 1,2, ..., n-1$}
  \State // Calculate the Givens matrix for 
  \State // zeroing out the element in $H(k+1,k)$ 
  \State $[c_k, s_k] \gets \textsc{calculate-givens}(H[k,k], H[k+1,k])$
  \State
  \State // Apply the Givens rotation from the left,
  \State // since $R=G_{k-1}G_{k-2} ... G_{2}G_{1}H$
  \State $ H \gets 
  			 \left[ \begin{array}{c c}
      			c_k & s_k \\
     			-s_k & c_k
  			 \end{array} \right]
		   H
		$
\EndFor
\State $R \gets H$
\State $Q \gets I$
\For{$k = 1,2, ..., n-1$}
  \State // Apply the Givens rotation from the right,
  \State // since $Q=G_{1}G_{2}...G_{k-1}$
  \State $ Q \gets Q 
  			\left[ \begin{array}{c c}
      			c_k & s_k \\
     			-s_k & c_k
  			\end{array} \right]
		$
\EndFor
\EndFunction
\end{algorithmic}

\subsubsection*{Sub-procedure for calculating the givens elements $s$ and $c$}
\begin{algorithmic}
\Function{calculate-givens}{x,y}
\State $r \gets \sqrt[]{x^2+y^2 }$ 
\State $c \gets \frac{x}{r} $
\State $s \gets \frac{y}{r} $
\State \Return $[c, s]$
\EndFunction
\end{algorithmic}

\subsubsection*{Number of flops for the $QR$-decomposition}
The $\textsc{calculate-givens}$ procedure contains one addition, two multiplications, two divisions, and one square root. 
Let $d$ be the number of flops for performing division, and let $r$ be the number of flops for performing a square root. 
In total $\textsc{calculate-givens}$ performs $3+2d+r$ flops. In $\textsc{compute-QR}$, $\textsc{calculate-givens}$ is performed $n-1$ times. 

Matrix multiplication with the Givens matrix is performed $n-1$ times twice in \textsc{compute-QR} (in both for-loops), but only the first $n-1$ times are necessary for transforming $H$ in upper triangular form. 
For every such Givens rotation, we have $6n$ flops \footnote{
It is here assumed that \textsc{matlab} carries out a matrix multiplication of a Givens matrix and any $n \times n$ matrix $A$ in such a manner that it applies the multiplication of the $2\times2$ Givens rotation with a $2\times n$ matrix extracted from $A$. 
All but 2 rows of the matrix $A$ will stay constant if a Givens rotation is applied, which is the basis of the assumption. 
The resulting matrix is a $2\times n$ matrix with $2n$ elements. 
Calculating each element is a scalar product which for this problem requires 3 operations. 
The total operation cost of the multiplication is therefore $2n\cdot 3=6n$. }

The total number of flops \footnote{$Q$ can then be calculated in $(n-1)6n$ more flops.} becomes $(n-1)6n+(n-1)(3+r+2d)$, a polynomial of order $O(n^2)$. \footnote{Note that a $QR$-decomposition with a full matrix would require a cost of $O(n^3)$, so we have gained a factor of $O(n)$ by doing the calculation in Hessenberg form.}

%\bibliosub
\end{document}