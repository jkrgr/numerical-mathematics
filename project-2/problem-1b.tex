%%%%%%%%%%%%%%%%%%%%%%%%%%%%%%%%%%%%%%
%
% This is how you write code:
%
% \begin{minted}{matlab}
% foo = [2 1 0;1 4 3;2 4.5 6];
% \end{minted}
%

% This is how you import code:
% 
% \inputminted[linenos]{matlab}{foo_bar.m}
%
 
% Most figures are imported this way:
%
% \begin{figure}
% \includegraphics[width=\textwidth]{foo_figure}
% \caption{This is a caption}
% \end{figure}
%
%%%%%%%%%%%%%%%%%%%%%%%%%%%%%%%%%%%%%%


\documentclass[00-main.tex]{subfiles}
\begin{document}



\subsection*{b.}
An upper Hessenberg matrix $H$ is a matrix whose elements below its first subdiagonal are zero, and an upper Hessenberg matrix is a matrix whose elements above its first superdiagonal is zero. $H$ is thus at most $n-1$ Givens rotations away from becoming a triangular matrix, since the number of elements in the first sub- or superdiagonal is $n-1$.

\subsubsection*{Procedure for computing the $QR$-decomposition of $H$ using Givens rotations}
\begin{algorithmic}
\If {$H$ is in upper triangular form}
\Comment{This is a comment}
  \State 
    do stuff
\ElsIf{$H$ is in upper triangular form}
  \State    
    \For{$i = 1 \to 10$} 
      \State $i \gets i + 1$
\EndFor
\EndIf
\end{algorithmic}

%\bibliosub
\end{document}