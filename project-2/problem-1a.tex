%%%%%%%%%%%%%%%%%%%%%%%%%%%%%%%%%%%%%%
%
% This is how you write code:
%
% \begin{minted}{matlab}
% foo = [2 1 0;1 4 3;2 4.5 6];
% \end{minted}
%

% This is how you import code:
% 
% \inputminted[linenos]{matlab}{foo_bar.m}
%
 
% Most figures are imported this way:
%
% \begin{figure}
% \includegraphics[width=\textwidth]{foo_figure}
% \caption{This is a caption}
% \end{figure}
%
%%%%%%%%%%%%%%%%%%%%%%%%%%%%%%%%%%%%%%


\documentclass[00-main.tex]{subfiles}
\begin{document}




\section*{Problem One}

\subsection*{a.}
We want to find the $QR$ decomposition of
\begin{equation}
H =
 \left[
 \begin{matrix}
    1 & 2 &  3 &  4 \\ 
    5 & 6 &  7 &  8 \\ 
    0 & 9 & 10 & 11 \\ 
    0 & 0 & 12 &  0
 \end{matrix} 
 \right]
\end{equation}

using Givens rotations. For every step the transformation should get closer to an upper triangular matrix. Here, three rotations are needed because $H$ would be an upper triangular matrix if $H(2,1)$, $H(3,2)$ and $H(4,3)$ was equal to zero. First $G_1$ is found so that $G_1 H(2,1) = 0$. Then $G_2$ is found so that $G_2 G_1 H(3,2) = 0$. At last $G_3$ is found so that $G_3 G_2 G_1 H(4,3) = 0$.
\begin{align*}
G_{1} &= 
 \left[
 \begin{matrix}
    0.1961  &  0.9806 & 0  &  0 \\
   -0.9806  &  0.1961 & 0  &  0 \\
    0       &  0      & 1  &  0 \\
    0       &  0      & 0  &  1
 \end{matrix}
 \right]
 \Rightarrow G_{1}H = 
  \left[
 \begin{matrix}
    5.0990  &  6.2757 & 7.4524  &  8.6291 \\
    0  &  -0.7845 & -1.5689  &  -2.3534 \\
    0       &  9      & 10  &  11 \\
    0       &  0      & 12  &  0
 \end{matrix}
 \right] \\
G_{2} &=
 \left[
 \begin{matrix}
    1 &  0      &   0      &  0 \\
    0 & -0.0868 &   0.9962 &  0 \\
    0 & -0.9962 &  -0.0868 &  0 \\
    0 &  0      &   0      &  1
 \end{matrix}
 \right]
  \Rightarrow G_2 G_1 H = 
  \left[
 \begin{matrix}
    5.0990  &  6.2757 & 7.4524   &  8.6291 \\
    0       &  9.0341 & 10.0985  &  11.1628 \\
    0       &  0      & 0.6947   &  1.3893 \\
    0       &  0      & 12       &  0
 \end{matrix}
 \right] \\
G_{3} &=
 \left[
 \begin{matrix}
    1  &  0  &  0      &  0      \\
    0  &  1  &  0      &  0      \\
    0  &  0  &  0.0578 &  0.9983 \\
    0  &  0  & -0.9983 &  0.0578 \\
 \end{matrix}
 \right]
  \Rightarrow G_3 G_2 G_1 H = 
  \left[
 \begin{matrix}
    5.0990  &  6.2757 & 7.4524   &  8.6291 \\
    0       &  9.0341 & 10.0985  &  11.1628 \\
    0       &  0      & 12.0201  &   0.0803 \\
    0       &  0      & 0       &  -1.3870
 \end{matrix}
 \right] 
\end{align*}

Now let $R = G_3 G_2 G_1 H$ and $Q^{-1} = G_3 G_2 G_1$. The $QR$-decoposition is then found by

\begin{align}
G_3 G_2 G_1 H &= R \\
Q^{-1} H &= R \\
H &= QR \\
\Rightarrow
Q = (G_3 G_2 G_1)^{-1} = G_1 G_2 G_3 &= 
  \left[
 \begin{matrix}
    0.1961  &  0.0851 &  0.0565   & -0.9752 \\
    0.9806  & -0.0170 & -0.0113  &  0.1950 \\
    0       &  0.9962 & -0.0050  &  0.0867 \\
    0       &  0      &  0.9983  &  0.0578
 \end{matrix}
 \right] \\
R &= 
  \left[
 \begin{matrix}
    5.0990  &  6.2757 & 7.4524   &  8.6291 \\
    0       &  9.0341 & 10.0985  &  11.1628 \\
    0       &  0      & 12.0201  &   0.0803 \\
    0       &  0      & 0       &  -1.3870
 \end{matrix}
 \right] 
\end{align}


%\bibliosub
\end{document}