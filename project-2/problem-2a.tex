%%%%%%%%%%%%%%%%%%%%%%%%%%%%%%%%%%%%%%
%
% This is how you write code:
%
% \begin{minted}{matlab}
% foo = [2 1 0;1 4 3;2 4.5 6];
% \end{minted}
%

% This is how you import code:
% 
% \inputminted[linenos]{matlab}{foo_bar.m}
%
 
% Most figures are imported this way:
%
% \begin{figure}
% \includegraphics[width=\textwidth]{foo_figure}
% \caption{This is a caption}
% \end{figure}
%
%%%%%%%%%%%%%%%%%%%%%%%%%%%%%%%%%%%%%%


\documentclass[00-main.tex]{subfiles}


\begin{document}

\section*{Problem Two}

\subsection*{a.}
The MatLab function eigroot in the attachments applies the built-in \textbf{eig} function in MatLab to factorize a given matrix A to the form $A=VDV^{\top}$, with the advantage that square rooting the diagonal matrix D requires only the square root of the individual diagonal elements.

The two iterative implementations described in Nicholas J. Highams paper Newton's method for the matrix square root[ref] are as follows:

\begin{equation}
(I): Y_{k+1}=\frac{1}{2}(Y_{k}+Y_{k}^{-1}A)
\end{equation}

\begin{align*}
(III):  &P_{k+1}=\frac{1}{2}(P_{k}+Q_{k}^{-1}) \\
          & Q_{k+1}=\frac{1}{2}(Q_{k}+P_{k}^{-1})
\end{align*}



In analyzing operation counts, one can see that iteration method (I): $Y_{k+1}=\frac{1}{2}(Y_{k}+Y_{k}^{-1}A$ includes $Y_{k}^{-1}A$. This is the solution of the linear system $Y_{k}x=A$. Using LU factorization, the operation count of solving such a system is $\frac{2}{3}n^{3}$ The other operations are matrix addition and multiplication of a constant, which are operations of the lower order $O(n^{2})$. Iteration method (III) includes the inversion of two matrices, each with an operation count of $n^{3}$, for a combined count of $2n^{3}$.

Figure 1 is a log-plot that visualizes the values in table ??. The easily identifiable parabola points to quadratic convergence. Theorem 2 in Newton's Method for the Matrix Square Root[ref] tells us that this is indeed the case. The figure also shows the linear divergence caused by the instability in the algorithm. 

As equation III[ref] shows, implementation (III) gives the same result of the iteration  $Y_k$ as implementation (I) with $Y_{0}=Q_{0}=I$. The same quadratic convergence is therefore visualized in Figure 2. When the norm of the error approaches the limit of accuracy for the computer program, in this case MatLab, the max norm diverges. The difference between implementation (I) and (III), will be that each iteration in implementation (III) will have a bounded effect on the max-norm, making it practically stable. The max-norm will remain almost constant in implementation (III), while it will diverge wildly in implementation (I). Looking at the condition numbers for the different matrices and the degree of divergence in implementation (III), one can see a correlation between the high condition numbers of $A_{100\times 100}$ and $A_{1000\times 1000}$ and the divergence in their respective tables. The matrices with lower condition numbers also diverge with implementation (III), but it's not evident with the limited iterations in their tables.�





%\bibliosub
\end{document}