%%%%%%%%%%%%%%%%%%%%%%%%%%%%%%%%%%%%%%
%
% This is how you write code:
%
% \begin{minted}{matlab}
% foo = [2 1 0;1 4 3;2 4.5 6];
% \end{minted}
%

% This is how you import code:
% 
% \inputminted[linenos]{matlab}{foo_bar.m}
%
 
% Most figures are imported this way:
%
% \begin{figure}
% \includegraphics[width=\textwidth]{foo_figure}
% \caption{This is a caption}
% \end{figure}
%
%%%%%%%%%%%%%%%%%%%%%%%%%%%%%%%%%%%%%%


\documentclass[00-main.tex]{subfiles}


\begin{document}

\section*{Problem Two}

\subsection*{a.}
The \textsc{Matlab} function eigroot in the attachments applies the built-in \textbf{eig} function in \textsc{Matlab} to factorize a given matrix A to the form $A=VDV^{\top}$, with the advantage that square rooting the diagonal matrix D requires only the square root of the individual diagonal elements.

The two iterative implementations described in Nicholas J. Highams paper: Newton's method for the matrix square root~\cite{paper} are as follows:

\begin{equation}
(I): Y_{k+1}=\frac{1}{2}(Y_{k}+Y_{k}^{-1}A)
\end{equation}

\begin{align}
(III):  &P_{k+1}=\frac{1}{2}(P_{k}+Q_{k}^{-1}) \\
          & Q_{k+1}=\frac{1}{2}(Q_{k}+P_{k}^{-1})
\label{eq: III}
\end{align}

These implementations were both derived from the iterations of Newton's method. Theorem 2 in Highams paper tells us that iteration method (I) converges quadratically with the identity matrix $I$ as the starting matrix $Y_0$. In method (III) the starting matrix is $Q_0=I$ with $P_0=A$. The relationship between the implementations being 
\begin{equation}
P_k=Y_k \\
\label{eq: relate2}
\end{equation}
\begin{equation}
Q_k=A^{-1}Y_k
\label{eq: relate}
\end{equation}

In analyzing operation counts, one can see that iteration method (I): $Y_{k+1}=\frac{1}{2}(Y_{k}+Y_{k}^{-1}A)$ includes $Y_{k}^{-1}A$. This is the solution of the linear system $Y_{k}x=A$. Using LU factorization, the operation count of solving such a system is $\frac{2}{3}n^{3}$ The other operations are matrix addition and multiplication of a constant, which are operations of the lower order, $O(n^{2})$. Iteration method (III) includes the inversion of two matrices, each with an operation count of $n^{3}$, for a combined count of $2n^{3}$. The operation counts for both implementations will therefore be of the order $O(n^3)$.

Figure \ref{fig:convergence plot I} is a log-plot that visualizes the values for the Wathen matrix in table \ref{lol}. The easily identifiable parabola on the log-plot points to quadratic convergence. All plots of the tables in the attachment are of this shape, and as mentioned, Theorem 2 in Newton's Method for the Matrix Square Root~\cite{paper} tells us that this is indeed the case. The figure also shows the linear divergence caused by the instability in the algorithm. 

As the relations in equations (\ref{eq: relate}) and (\ref{eq: relate2})  shows, implementation (III) gives the same result of the iteration  $Y_k$ as implementation (I) with $Y_{0}=Q_{0}=I$. The same quadratic convergence is therefore visualized in Figure 2. When the norm of the error approaches the limit of accuracy for the computer program, in this case \textsc{Matlab}, the max norm diverges, but slowly. The difference between implementation (I) and (III), will be that each iteration in implementation (III) will have a bounded effect on the max-norm, making it practically stable. The max-norm will remain almost constant in implementation (III), while it will diverge rapidly in implementation (I). Looking at the condition numbers in table \ref{tull} for the different matrices and the degree of divergence in implementation (III), one can see a correlation between the high condition numbers and the rate of convergence in their respective tables.


\begin{thebibliography}{00}
\bibitem{paper} Nicholas J. Higham \textit{Newton's Method for the Matrix Square Root} MATHEMATICS COMPUTATION VOLUME 46, NUMBER 174, APRIL 1986, Pages 537-549
\end{thebibliography}

%\bibliosub
\end{document}