%%%%%%%%%%%%%%%%%%%%%%%%%%%%%%%%%%%%%%
%
% This is how you write code:
%
% \begin{minted}{matlab}
% foo = [2 1 0;1 4 3;2 4.5 6];
% \end{minted}
%

% This is how you import code:
% 
% \inputminted[linenos]{matlab}{foo_bar.m}
%
 
% (Most) figures are imported this way:
%
% \begin{figure}
% \includegraphics[width=\textwidth]{foo_figure}
% \caption{This is a caption}
% \end{figure}
%
%%%%%%%%%%%%%%%%%%%%%%%%%%%%%%%%%%%%%%


\documentclass[00-main.tex]{subfiles}
\begin{document}


\subsection*{b.}
We want to find the partial derivative and the gradient of $\phi$ given by the function

\begin{equation}
\label{eq: Phi 1}
\phi \left(W\right) 
= \pm \frac{1}{4} \displaystyle\sum\limits_{i=1}^n \text{kurt}(\mathbf{w}_{i}^{T} \tilde{x})
= \pm \frac{1}{4} \left( \displaystyle\sum\limits_{i=1}^n E\left[(\mathbf{w}_{i}^{T} \tilde{x})^4\right] - 3n\right).
\end{equation}

$W$ have columns $\mathbf{w_1,\cdots,w_n}$ and entries ${w_{i,j}, 1 \leq i,j \leq n}$.
 $\phi$ is a measure for the non-Gaussianity, and is measured in kurtosis which is defined in scalar as

\begin{equation}
\label{eq: Kurtosis 1}
\text{kurt}(z):= E[z^4]-3E[z^2].
\end{equation}
Further on, if $z$ has unitary variance the kortosis is computed as

\begin{equation}
\label{eq: Kurtosis 2}
\text{kurt}(z):= E[z^4]-3.
\end{equation}

If we write out equation (\ref{eq: Phi 1}) we end ut with

\begin{equation}
\label{eq:Phi 3}
\phi \left(W\right)
= \pm \frac{1}{4} \left( E\left[ \left( w_{1,1}^{T}\tilde{x}_1+\cdots +w_{1,n}^{T}\tilde{x}_{n}\right)^4\right]
+\cdots + E\left[\left(w_{n,1}^{T}\tilde{x}_1+\cdots +w_{n,n}^{T}\tilde{x}_{n}\right)^4\right]-3n\right).
\end{equation}

Where $w_{i,j}^{T}$ is the entries for $W^{T}$. If we express the equation in sums we can write it like

\begin{equation}
\label{eq: Phi 4}
\begin{split}
&\phi \left(W\right)
 = \pm \frac{1}{4} \left( \displaystyle\sum\limits_{i=1}^n E\left(\displaystyle\sum\limits_{j=1}^n \left( w_{i,j}^{T} \tilde{x}_{j} \right)^{4} \right)-3n\right)\\
& = \pm \frac{1}{4} \left( \displaystyle\sum\limits_{i=1}^n E\left(\displaystyle\sum\limits_{j=1}^n \left( w_{j,i} \tilde{x}_{j} \right)^{4} \right)-3n\right)\\
& =\pm \frac{1}{4} \left( \displaystyle\sum\limits_{j=1}^n E\left(\displaystyle\sum\limits_{i=1}^n \left( w_{i,j} \tilde{x}_{i} \right)^{4} \right)-3n\right).\\
\end{split}
\end{equation}

With $y(t):=W^{T}\tilde{x}(t)$, the partial derivative of $\phi$ can the be written as

\begin{equation}
\label{eq: PD 1}
\begin{split}
& \frac{\partial \phi}{\partial w_{i,j}} = \pm \frac{1}{4} \left(E\left(4\tilde{x_i}\left(w_{i,j}\tilde{x_i}\right)^3\right)\right)\\
& =\pm \left(E\left(\tilde{x_i}\left(w_{j,i}^{T}\tilde{x_i}\right)^3\right)\right)\\
& = \pm E\left(\tilde{x_i}\left(y_{j}\right)^3\right).
\end{split}
\end{equation}

By combining the partial deriviatives in a $n \times n$ matrix, we se that the gradient of $\phi$ is given by
\begin{equation}
\label{eq: gardient}
\frac{\partial \phi}{\partial W}= \pm E\left[\tilde{x}\left(y.^{3}\right)^{T}\right].
\end{equation}

Where $y.^3$ is the vector whose components are cubes of the components of $y$.

\newpage

%\bibliosub
\end{document}