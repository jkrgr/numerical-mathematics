

%%%%%%%%%%%%%%%%%%%%%%%%%%%%%%%%%%%%%%
%
% This is how you write code:
%
% \begin{minted}{matlab}
% foo = [2 1 0;1 4 3;2 4.5 6];
% \end{minted}
%

% This is how you import code:
% 
% \inputminted[linenos]{matlab}{foo_bar.m}
%
 
% Most figures are imported this way:
%
% \begin{figure}
% \includegraphics[width=\textwidth]{foo_figure}
% \caption{This is a caption}
% \end{figure}
%
%%%%%%%%%%%%%%%%%%%%%%%%%%%%%%%%%%%%%%


\documentclass[00-main.tex]{subfiles}
\begin{document}

\section*{Problem 1}

\subsection*{1a)}

Let $W=W(\tau)$, and denote with $\dot{W}$ the deriviative with respect to $\tau$. Given the system of differential equations 

\begin{equation}
\label{eq: Grad(W)}
\dot{W}=-\text{grad}(W), \: \; \: \text{grad}(W):=(\frac{\partial \phi}{\partial W}W^{T} - W\frac{\partial \phi}{\partial W}^{T})W.
\end{equation}

In the following tasks lets consider$\frac{\partial \phi}{\partial W}=A$.
First verify that if $W(0)^{T}W(0)=I$ then $W(\phi)^{T}W(\phi)=I$, for all $ \phi \geq 0$. To verify this consider

\begin{equation}
\label{eq: integration 1}
\int _{0}^{\tau} \frac{d}{ds} (W(s)^{T}W(s)) ds = W(\phi)^{T}W(\phi) - W(0)^{T}W(0).
\end{equation}

Lets $\dot{W}$ equal $S(W)W$, where $S(W)$ is a skew-symmetric matrix.
 Where $\frac{d}{ds} (W(s)^{T}W(s))$ can be written out as the following from equtaion(\ref{eq: Grad(W)}).
 
 
 \begin{equation}
 \label{eq: Div WtW}
 \begin{split}
 &\frac{d}{ds} (W(s)^{T}W(s))=\dot{W^{T}}W+W^{T}\dot{W}=W^{T}(-\text{grad}(W))+(-\text{grad}(W))^{T}W \\
 &=W^{T}(AW^{T}-WA^{T})W+W^{T}(WA^T-AW^T)W\\
 &=W^{T}(AW^{T}-WA^{T})W-W^{T}(AW^T-WA^{T})W\\
 &=W^{T}S(W)W-W^{T}S(W)W=0.
 \end{split}
 \end{equation}
 
 We thereby have from equation(\ref{eq: integration 1})
 \begin{equation}
 \label{eq: Verifiy 1}
 \begin{split}
 &\int _{0}^{\tau} \frac{d}{ds} (W(s)^{T}W(s)) ds = W(\phi)^{T}W(\phi) - W(0)^{T}W(0) = 0 \\
 &\Longrightarrow W(\phi)^{T}W(\phi) = W(0)^{T}W(0).
\end{split} 
 \end{equation}
 
 Lets now consider the function $\gamma(\tau)$
 
 \begin{equation}
 \label{eq: gamma}
 \gamma(\tau):=\phi(W(\tau)),
 \end{equation}
 
 and show that $\gamma(\tau_1) \leq \gamma(\tau_0)$ if $\tau_1 \geq \tau_0$. The derivative of $\gamma$ can be expressed as
 
 \begin{equation}
 \label{eq: div gamma}
 \dot{\gamma}(\tau)=\text{trace}(\frac{\partial \phi}{\partial W}^{T} \dot{W}(\tau)).
 \end{equation}
 
 Where 'trace' indicates the trace of the matrix, the sum of the diagonal elements in the matrix.
 
 First we show that $\dot{\gamma(\tau)}\leq 0$, and we let$W^{T}W=I$.
 
 \begin{equation}
 \label{eq: Verify 2}
 \begin{split}
 &\text{By using the following trace-properties, and Cauchy-Schwarz inequality for innerproduct:}\\
 & 1)\mid <A,B> \mid ^{2} \leq  \:<A,A>*<B,B>, \: \text{where} <A,B>=\text{trace}(A^{*}B) \\
 & 2)\: \text{trace}(A^{T}B)\leq\sqrt{\text{trace}(A^{T}A)}\sqrt{\text{trace}(B^{T}B}) \\
 & 3)\: \text{trace}(AB)=\text{trace}(BA), \text{the trace is invariant under cyclic permutations, i.e}\\
 & \text{trace}(ABCD)=\text{trace}(ACDB)=\text{trace}(ADCB)\\
 & 4)\: \text{trace}(A+B)=\text{trace}(A)+\text{trace}(B)\\
 &\text{We get the following expression}\\
 &\dot{\gamma}(\tau)=\text{trace}(\frac{\partial \phi}{\partial W}^{T} \dot{W}(\tau))\\
 &= \text{trace}(-A^{T}(AW^{T} -WA^{T})W)\\
 &= \text{trace}(-A^{T}AW^{T}W+A^{T}WA^{T}W)\\
 &= \text{trace}(A^{T}WA^{T}W-A^{T}AW^{T}W) \\
 &= \text{trace}(A^{T}WA^{T}W) -\text{trace}(A^{T}AW^{T}W)\\
 &\leq \sqrt{\text{trace}(A^{T}A)} \sqrt{\text{trace}(W^{T}AW^{T}WA^{T}W)} -\text{trace}(A^{T}AW^{T}W)\\
 &=\sqrt{\text{trace}(A^{T}A)} \sqrt{\text{trace}(W^{T}AIA^{T}W)} -\text{trace}(A^{T}AI)\\
 &=\sqrt{\text{trace}(A^{T}A)} \sqrt{\text{trace}(A^{T}AW^{T}W)} -\text{trace}(A^{T}A)\\
 &=\sqrt{\text{trace}(A^{T}A)} \sqrt{\text{trace}(A^{T}AI)} -\text{trace}(A^{T}A)\\
 &=\sqrt{\text{trace}(A^{T}A)} \sqrt{\text{trace}(A^{T}A)} -\text{trace}(A^{T}A)\\
 &=\text{trace}(A^{T}A) -\text{trace}(A^{T}A)=0\: , \: \Longrightarrow \dot{\gamma}(\tau) \leq 0.
 \end{split}
 \end{equation}
 
 Since $\dot{\gamma}(\tau)\leq 0$ for all values of $\tau$ it is thereby given that $\gamma(\tau_1) \leq \gamma(\tau_0)$ if $\tau_1 \geq \tau_0$ since $\gamma$ is deacreasing when $\tau$ is inreasing.
 
