

%%%%%%%%%%%%%%%%%%%%%%%%%%%%%%%%%%%%%%
%
% This is how you write code:
%
% \begin{minted}{matlab}
% foo = [2 1 0;1 4 3;2 4.5 6];
% \end{minted}
%

% This is how you import code:
% 
% \inputminted[linenos]{matlab}{foo_bar.m}
%
 
% Most figures are imported this way:
%
% \begin{figure}
% \includegraphics[width=\textwidth]{foo_figure}
% \caption{This is a caption}
% \end{figure}
%
%%%%%%%%%%%%%%%%%%%%%%%%%%%%%%%%%%%%%%


\documentclass[00-main.tex]{subfiles}
\begin{document}
\section*{Problem 1}
\subsection*{1a)}

We will here present an approximate solution to the cocktail party problem using Independent Component Analysis (ICA). 
Our starting point is the following system of differential equations,

\begin{equation}
\label{eq: Grad(W)}
\dot{W}=-\text{grad}(W), \: \; \: \text{grad}(W):=(\frac{\partial \phi}{\partial W}W^{T} - W\frac{\partial \phi}{\partial W}^{T})W.
\end{equation}

Here $W=W(\tau)$ and the deriviative of $W$ with respect to $\tau$ is $\dot{W}$. 
%In the following tasks consider 
To ease the calculations let $\frac{\partial \phi}{\partial W}=A$.
%First verify that if $W(0)^{T}W(0)=I$ then $W(\phi)^{T}W(\phi)=I$, for all $ \phi \geq 0$. To verify this consider
We will here show that

\begin{equation}
\label{show-1a}
W(0)^{T}W(0)=I \implies W(\phi)^{T}W(\phi)=I, \phi \geq 0
\end{equation}

holds.

Consider the integral

\begin{equation}
\label{eq: integration 1}
\int _{0}^{\tau} \frac{d}{ds} (W(s)^{T}W(s)) ds = W(\phi)^{T}W(\phi) - W(0)^{T}W(0).
\end{equation}

and let $\dot{W}$ equal $S(W)W$, where $S(W)$ is a skew-symmetric matrix.
By \cref{eq: Grad(W)} $\frac{d}{ds} \left(W(s)^{T}W(s)\right)$ can be written out as 
 
 
\begin{equation}
 \label{eq: Div WtW}
 \begin{split}
  &\frac{d}{ds} (W(s)^{T}W(s))=\dot{W^{T}}W+W^{T}\dot{W}=W^{T}(-\text{grad}(W))+(-\text{grad}(W))^{T}W \\
  &=W^{T}(AW^{T}-WA^{T})W+W^{T}(WA^T-AW^T)W\\
  &=W^{T}(AW^{T}-WA^{T})W-W^{T}(AW^T-WA^{T})W\\
  &=W^{T}S(W)W-W^{T}S(W)W=0.
 \end{split}
\end{equation}

So that \cref{eq: integration 1} can be used to get
 
\begin{equation}
 \label{eq: Verifiy 1}
 \begin{split}
    &\int _{0}^{\tau} \frac{d}{ds} (W(s)^{T}W(s)) ds = W(\phi)^{T}W(\phi) - W(0)^{T}W(0) = 0 \\
    &\Longrightarrow W(\phi)^{T}W(\phi) = W(0)^{T}W(0).
  \end{split}
\end{equation}
 
Consider now the function $\gamma(\tau)$
 
 \begin{equation}
 \label{eq: gamma}
 \gamma(\tau):=\phi(W(\tau)),
 \end{equation}
 
which we will show to satisfy $\gamma(\tau_1) \leq \gamma(\tau_0)$ if $\tau_1 \geq \tau_0$. 

The derivative of $\gamma$ can be expressed as
 
  \label{eq: div gamma}
 \begin{equation}
  \dot{\gamma}(\tau)=\text{tr}(\frac{\partial \phi}{\partial W}^{T} \dot{W}(\tau)).
 \end{equation}
 
 Where 'tr' indicates the trace of the matrix, i.e. the sum of the diagonal elements in the matrix.
 
 While letting $W^{T}W=I$, it can be shown that $\dot{\gamma(\tau)} \leq 0$. 
 The Cauchy-Schwarz inequality for inner product (\cref{cauchy}), and the trace properties (\crefrange{trace1}{trace4}),
 
\begin{align}
  |<A,B>|^{2}       &\leq \:<A,A>*<B,B>, \qquad <A,B> = \text{tr}(A^{*}B) \label{cauchy} \\
  \text{tr}(A^{T}B) &\leq \sqrt{\text{tr}(A^{T}A)}\sqrt{\text{tr}(B^{T}B}) \label{trace1} \\
  \text{tr}(AB)     &= \text{tr}(BA) \\
  \text{tr}(ABCD)   &= \text{tr}(ACDB) = \text{tr}(ADCB) \\  
  \text{tr}(A+B)    &= \text{tr}(A) + \text{tr}(B) \label{trace4}
\end{align}

- and some hand crafting, 


\begin{align*}
\dot{\gamma}(\tau)=\text{trace}(\frac{\partial \phi}{\partial W}^{T} \dot{W}(\tau))\\
    \text{trace}(-A^{T}(AW^{T} -WA^{T})W) = \text{trace}(-A^{T}AW^{T}W+A^{T}WA^{T}W)\\
    = \text{trace}(A^{T}WA^{T}W-A^{T}AW^{T}W) 
    = \text{trace}(A^{T}WA^{T}W) -\text{trace}(A^{T}AW^{T}W)\\
    \leq \sqrt{\text{trace}(A^{T}A)} \sqrt{\text{trace}(W^{T}AW^{T}WA^{T}W)} -\text{trace}(A^{T}AW^{T}W)\\
    = \sqrt{\text{trace}(A^{T}A)} \sqrt{\text{trace}(W^{T}AIA^{T}W)} -\text{trace}(A^{T}AI)\\
    = \sqrt{\text{trace}(A^{T}A)} \sqrt{\text{trace}(A^{T}AW^{T}W)} -\text{trace}(A^{T}A)\\
    = \sqrt{\text{trace}(A^{T}A)} \sqrt{\text{trace}(A^{T}AI)} -\text{trace}(A^{T}A)\\
    = \sqrt{\text{trace}(A^{T}A)} \sqrt{\text{trace}(A^{T}A)} -\text{trace}(A^{T}A)\\
    = \text{trace}(A^{T}A) -\text{trace}(A^{T}A)=0 \\ 
    \Longrightarrow \dot{\gamma}(\tau) \leq 0.
\end{align*}
 
 provide the desired result. $\dot{\gamma}(\tau)\leq 0$ for all values of $\tau$ so that $\gamma(\tau_1) \leq \gamma(\tau_0)$ if $\tau_1 \geq \tau_0$ since $\gamma$ is decreasing when $\tau$ is increasing.
 
% \begin{equation}
%     \label{eq: Verify 2}
%     \begin{split}
%         &\text{By using the following trace-properties, and Cauchy-Schwarz inequality for innerproduct:}\\
%         & 1) | <A,B> | ^{2} \leq  \:<A,A>*<B,B>, \: \text{where} <A,B>=\text{trace}(A^{*}B) \\
%         & 2)\: \text{trace}(A^{T}B)\leq\sqrt{\text{trace}(A^{T}A)}\sqrt{\text{trace}(B^{T}B}) \\
%         & 3)\: \text{trace}(AB)=\text{trace}(BA), \text{the trace is invariant under cyclic permutations, i.e}\\
%         & \text{trace}(ABCD)=\text{trace}(ACDB)=\text{trace}(ADCB)\\
%         & 4)\: \text{trace}(A+B)=\text{trace}(A)+\text{trace}(B)\\
%         &\text{We get the following expression}\\
%         &\dot{\gamma}(\tau)=\text{trace}(\frac{\partial \phi}{\partial W}^{T} \dot{W}(\tau))\\
%         &= \text{trace}(-A^{T}(AW^{T} -WA^{T})W)\\
%         &= \text{trace}(-A^{T}AW^{T}W+A^{T}WA^{T}W)\\
%         &= \text{trace}(A^{T}WA^{T}W-A^{T}AW^{T}W) \\
%         &= \text{trace}(A^{T}WA^{T}W) -\text{trace}(A^{T}AW^{T}W)\\
%         &\leq \sqrt{\text{trace}(A^{T}A)} \sqrt{\text{trace}(W^{T}AW^{T}WA^{T}W)} -\text{trace}(A^{T}AW^{T}W)\\
%         &=\sqrt{\text{trace}(A^{T}A)} \sqrt{\text{trace}(W^{T}AIA^{T}W)} -\text{trace}(A^{T}AI)\\
%         &=\sqrt{\text{trace}(A^{T}A)} \sqrt{\text{trace}(A^{T}AW^{T}W)} -\text{trace}(A^{T}A)\\
%         &=\sqrt{\text{trace}(A^{T}A)} \sqrt{\text{trace}(A^{T}AI)} -\text{trace}(A^{T}A)\\
%         &=\sqrt{\text{trace}(A^{T}A)} \sqrt{\text{trace}(A^{T}A)} -\text{trace}(A^{T}A)\\
%         &=\text{trace}(A^{T}A) -\text{trace}(A^{T}A)=0\: , \: \Longrightarrow \dot{\gamma}(\tau) \leq 0.
%     \end{split}
% \end{equation}
 
% Since $\dot{\gamma}(\tau)\leq 0$ for all values of $\tau$ it is thereby given that $\gamma(\tau_1) \leq \gamma(\tau_0)$ if $\tau_1 \geq \tau_0$ since $\gamma$ is decreasing when $\tau$ is increasing.
 
\end{document}
