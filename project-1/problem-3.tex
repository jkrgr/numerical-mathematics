\documentclass[00-main.tex]{subfiles}
\begin{document}

\section*{Problem Three}

\subsection*{a.}
Let $\mathbf{v} \neq 0$. Each row of $\mathbf{uv}^T$ is a multiple of $\mathbf{v}^T$ so the row space of $\mathbf{uv}^T$ has dimension 1. This implies that the rank of the matrix is 1.

\subsection*{b.}

\begin{align}
\nonumber
I = AA^{-1} = (I-\mathbf{uv}^T)(I+\gamma \mathbf{uv}^T) &= I + \gamma \mathbf{uv}^T - \mathbf{uv}^T - \mathbf{uv}^T \gamma \mathbf{uv}^T \\
\implies
( \gamma - 1) \mathbf{uv}^T + \gamma \mathbf{u} ( \mathbf{v}^T \mathbf{u}) \mathbf{v}^T &= 
( \gamma - 1) \mathbf{uv}^T + ( \mathbf{v}^T \mathbf{u}) \gamma \mathbf{u} \mathbf{v}^T
= 0
\end{align}

Since both $\gamma$ and the inner product of $\mathbf{u}$ and $\mathbf{v}$, $\mathbf{u}^T \mathbf{v}$, are scalars. Dividing by $\mathbf{uv}^T$ and solving for $\gamma$ yields

\begin{align}
\nonumber
( \gamma - 1) + ( \mathbf{v}^T \mathbf{u}) \gamma &= 0 \\
\implies
\gamma = \frac{1}{1+ \mathbf{v}^T \mathbf{u}}
\end{align}

\subsection*{c.}
As $\mathbf{v}^T \mathbf{u}$ approaches one, $\gamma$ approaches infinity. In that case, the matrix $A$ is singular and thus not invertible.

\subsection*{d.}
$\kappa_2(A) = ||A||_2 ||A^{-1}||_2 = ||I-\mathbf{uv}^T||_2 ||I+\gamma \mathbf{uv}^T||_2 $

An upper bound for $\kappa_2(A)$ can be found by using the triangle inequality.

\begin{align*}
\kappa_2(A) & \leq (||I||_2-||\mathbf{uv}^T||_2) (||I||_2+||\gamma \mathbf{uv}^T||_2) \\
&= (||I||_2-||\mathbf{uv}^T||_2) (||I||_2+||\gamma \mathbf{uv}^T||_2)
\end{align*}




%\bibliosub
\end{document}