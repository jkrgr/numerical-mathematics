\documentclass[00-00-main.tex]{subfiles}
\begin{document}

\section{Problem One}
T. Mitchell defines a well posed learning problem as
\begin{listing}
\textit{A computer program is said to learn from experience E with respect to some
class of tasks T and performance measure P, if its performance at tasks in T,
as measured by P, improve by experience E.}
\end{listing}
In mathematics a well posed problem is one that has a (unique) solution, and where a
small change in the problem's initial condition(s) hardly affects the solution.

\subsection{Pregnant customers}
A large food chain would like to predict whether a customer is pregnant or not, based
solely on the products that is bought. Using the definition above, we can describe this
as a well posed learning problem.

\item Hello
\item You



\section{Coding}

\begin{lstlisting}
def foo(bar):
    print 'Eat my ' + bar
\end{lstlisting}

% listing makes a floating box with top and bottom borders
\begin{listing}
% minted for syntax highlighting
\begin{minted}[linenos]{python} 
def foo(bar='Chocolate'):
    rage = 'Did you eat my ' + bar + '?!'
    print rage
\end{minted}
\end{listing}
Remember to \mint{python} |import gravity|

\section{Citing}
This is a citation\cite{kiss-2010-I}. And you can use the citation verbally as well, according to \citet{kiss-2010-I}.

\section{Mathemagic}
An inline equation is written $x^3 = 8$. You can also use this
\begin{equation}
\label{eq:1337-equation}
x^5+4x^2=1337x
\end{equation}
to write a 1337 equation.
\bibliosub
\end{document}