\documentclass[00-main.tex]{subfiles}
\begin{document}

\section*{Problem One}

\subsection*{a.}

\begin{minted}{matlab}
>> A = [2 1 0;1 4 3;2 4.5 6];
>> b = [2 0 -3.5]';
>> x = A\b
x =
   0.50000
   1.00000
  -1.50000
>> % Checking the answer:
>> A*x
ans =
   2.00000
   0.00000
  -3.50000
>> % Answer is b, as expected.
\end{minted}

\subsection*{b.}

\begin{minted}{matlab}
>> % Finding the LU-decomposition of A where L is the
>> % lower- and U is the upper triangular matrix.
>> [L U] = lu(A);
>> L
L =
   1.00000   0.00000   0.00000
   0.50000   1.00000   0.00000
   1.00000   1.00000   1.00000
>> U
U =
   2.00000   1.00000   0.00000
   0.00000   3.50000   3.00000
   0.00000   0.00000   3.00000
\end{minted}

\subsection*{c.}

\inputminted[linenos]{matlab}{newton_c.m}

\subsection*{d.}

\inputminted[linenos]{matlab}{newton_d.m}

\subsection*{e.}
The system of the two coupled first order ODEs and can be represented as a column vector of state derivatives, $ {d \overrightarrow{y} \over dt} = [{dy_1 \over dt}, {dy_2 \over dt}]^T $. 
An (anonymous) function that returns such a column vector is passed to $euler$, and the ODEs are then solved simultaneously.

\begin{minted}{matlab}
>> [t y] = euler(@(t,y) [-sin(y(2));y(1)], 0, 10, [1.5;0], 0.1);
>> plot(t, y(1,:), t, y(2,:));
>> xlabel('t');
>> legend('y_1', 'y_2');
\end{minted}

\begin{figure}
\includegraphics[width=\textwidth]{euler_plot}
\caption{Plot shows $y_1$ and $y_2$ that was found through Euler's method.}
\end{figure}


%\bibliosub
\end{document}